\documentclass[]{article}
\usepackage{lmodern}
\usepackage{amssymb,amsmath}
\usepackage{ifxetex,ifluatex}
\usepackage{fixltx2e} % provides \textsubscript
\ifnum 0\ifxetex 1\fi\ifluatex 1\fi=0 % if pdftex
  \usepackage[T1]{fontenc}
  \usepackage[utf8]{inputenc}
\else % if luatex or xelatex
  \ifxetex
    \usepackage{mathspec}
  \else
    \usepackage{fontspec}
  \fi
  \defaultfontfeatures{Ligatures=TeX,Scale=MatchLowercase}
\fi
% use upquote if available, for straight quotes in verbatim environments
\IfFileExists{upquote.sty}{\usepackage{upquote}}{}
% use microtype if available
\IfFileExists{microtype.sty}{%
\usepackage{microtype}
\UseMicrotypeSet[protrusion]{basicmath} % disable protrusion for tt fonts
}{}
\usepackage[margin=1in]{geometry}
\usepackage{hyperref}
\hypersetup{unicode=true,
            pdftitle={Report},
            pdfauthor={pushpita panigrahi(pxp171530), akash chand(axc173730), siddharth swarup panda(ssp171730)},
            pdfborder={0 0 0},
            breaklinks=true}
\urlstyle{same}  % don't use monospace font for urls
\usepackage{color}
\usepackage{fancyvrb}
\newcommand{\VerbBar}{|}
\newcommand{\VERB}{\Verb[commandchars=\\\{\}]}
\DefineVerbatimEnvironment{Highlighting}{Verbatim}{commandchars=\\\{\}}
% Add ',fontsize=\small' for more characters per line
\usepackage{framed}
\definecolor{shadecolor}{RGB}{248,248,248}
\newenvironment{Shaded}{\begin{snugshade}}{\end{snugshade}}
\newcommand{\KeywordTok}[1]{\textcolor[rgb]{0.13,0.29,0.53}{\textbf{#1}}}
\newcommand{\DataTypeTok}[1]{\textcolor[rgb]{0.13,0.29,0.53}{#1}}
\newcommand{\DecValTok}[1]{\textcolor[rgb]{0.00,0.00,0.81}{#1}}
\newcommand{\BaseNTok}[1]{\textcolor[rgb]{0.00,0.00,0.81}{#1}}
\newcommand{\FloatTok}[1]{\textcolor[rgb]{0.00,0.00,0.81}{#1}}
\newcommand{\ConstantTok}[1]{\textcolor[rgb]{0.00,0.00,0.00}{#1}}
\newcommand{\CharTok}[1]{\textcolor[rgb]{0.31,0.60,0.02}{#1}}
\newcommand{\SpecialCharTok}[1]{\textcolor[rgb]{0.00,0.00,0.00}{#1}}
\newcommand{\StringTok}[1]{\textcolor[rgb]{0.31,0.60,0.02}{#1}}
\newcommand{\VerbatimStringTok}[1]{\textcolor[rgb]{0.31,0.60,0.02}{#1}}
\newcommand{\SpecialStringTok}[1]{\textcolor[rgb]{0.31,0.60,0.02}{#1}}
\newcommand{\ImportTok}[1]{#1}
\newcommand{\CommentTok}[1]{\textcolor[rgb]{0.56,0.35,0.01}{\textit{#1}}}
\newcommand{\DocumentationTok}[1]{\textcolor[rgb]{0.56,0.35,0.01}{\textbf{\textit{#1}}}}
\newcommand{\AnnotationTok}[1]{\textcolor[rgb]{0.56,0.35,0.01}{\textbf{\textit{#1}}}}
\newcommand{\CommentVarTok}[1]{\textcolor[rgb]{0.56,0.35,0.01}{\textbf{\textit{#1}}}}
\newcommand{\OtherTok}[1]{\textcolor[rgb]{0.56,0.35,0.01}{#1}}
\newcommand{\FunctionTok}[1]{\textcolor[rgb]{0.00,0.00,0.00}{#1}}
\newcommand{\VariableTok}[1]{\textcolor[rgb]{0.00,0.00,0.00}{#1}}
\newcommand{\ControlFlowTok}[1]{\textcolor[rgb]{0.13,0.29,0.53}{\textbf{#1}}}
\newcommand{\OperatorTok}[1]{\textcolor[rgb]{0.81,0.36,0.00}{\textbf{#1}}}
\newcommand{\BuiltInTok}[1]{#1}
\newcommand{\ExtensionTok}[1]{#1}
\newcommand{\PreprocessorTok}[1]{\textcolor[rgb]{0.56,0.35,0.01}{\textit{#1}}}
\newcommand{\AttributeTok}[1]{\textcolor[rgb]{0.77,0.63,0.00}{#1}}
\newcommand{\RegionMarkerTok}[1]{#1}
\newcommand{\InformationTok}[1]{\textcolor[rgb]{0.56,0.35,0.01}{\textbf{\textit{#1}}}}
\newcommand{\WarningTok}[1]{\textcolor[rgb]{0.56,0.35,0.01}{\textbf{\textit{#1}}}}
\newcommand{\AlertTok}[1]{\textcolor[rgb]{0.94,0.16,0.16}{#1}}
\newcommand{\ErrorTok}[1]{\textcolor[rgb]{0.64,0.00,0.00}{\textbf{#1}}}
\newcommand{\NormalTok}[1]{#1}
\usepackage{graphicx,grffile}
\makeatletter
\def\maxwidth{\ifdim\Gin@nat@width>\linewidth\linewidth\else\Gin@nat@width\fi}
\def\maxheight{\ifdim\Gin@nat@height>\textheight\textheight\else\Gin@nat@height\fi}
\makeatother
% Scale images if necessary, so that they will not overflow the page
% margins by default, and it is still possible to overwrite the defaults
% using explicit options in \includegraphics[width, height, ...]{}
\setkeys{Gin}{width=\maxwidth,height=\maxheight,keepaspectratio}
\IfFileExists{parskip.sty}{%
\usepackage{parskip}
}{% else
\setlength{\parindent}{0pt}
\setlength{\parskip}{6pt plus 2pt minus 1pt}
}
\setlength{\emergencystretch}{3em}  % prevent overfull lines
\providecommand{\tightlist}{%
  \setlength{\itemsep}{0pt}\setlength{\parskip}{0pt}}
\setcounter{secnumdepth}{0}
% Redefines (sub)paragraphs to behave more like sections
\ifx\paragraph\undefined\else
\let\oldparagraph\paragraph
\renewcommand{\paragraph}[1]{\oldparagraph{#1}\mbox{}}
\fi
\ifx\subparagraph\undefined\else
\let\oldsubparagraph\subparagraph
\renewcommand{\subparagraph}[1]{\oldsubparagraph{#1}\mbox{}}
\fi

%%% Use protect on footnotes to avoid problems with footnotes in titles
\let\rmarkdownfootnote\footnote%
\def\footnote{\protect\rmarkdownfootnote}

%%% Change title format to be more compact
\usepackage{titling}

% Create subtitle command for use in maketitle
\newcommand{\subtitle}[1]{
  \posttitle{
    \begin{center}\large#1\end{center}
    }
}

\setlength{\droptitle}{-2em}

  \title{Report}
    \pretitle{\vspace{\droptitle}\centering\huge}
  \posttitle{\par}
    \author{pushpita panigrahi(pxp171530), akash chand(axc173730), siddharth swarup
panda(ssp171730)}
    \preauthor{\centering\large\emph}
  \postauthor{\par}
    \date{}
    \predate{}\postdate{}
  

\begin{document}
\maketitle

\subsection{Read the token data}\label{read-the-token-data}

We chose networkbnbTX token as our dataset.

\begin{Shaded}
\begin{Highlighting}[]
\NormalTok{file <-}\StringTok{'networkbnbTX.txt'}
\NormalTok{col_names <-}\StringTok{ }\KeywordTok{c}\NormalTok{(}\StringTok{"FROMNODE"}\NormalTok{,}\StringTok{"TONODE"}\NormalTok{,}\StringTok{"DATE"}\NormalTok{,}\StringTok{"TOKENAMOUNT"}\NormalTok{)}
\NormalTok{mydata <-}\StringTok{ }\KeywordTok{read.csv}\NormalTok{( file, }\DataTypeTok{header =} \OtherTok{FALSE}\NormalTok{, }\DataTypeTok{sep =} \StringTok{" "}\NormalTok{, }\DataTypeTok{dec =} \StringTok{"."}\NormalTok{, }\DataTypeTok{col.names =}\NormalTok{ col_names)}
\NormalTok{mydata}\OperatorTok{$}\NormalTok{DATE <-}\StringTok{ }\KeywordTok{as.Date}\NormalTok{(}\KeywordTok{as.POSIXct}\NormalTok{(}\KeywordTok{as.numeric}\NormalTok{(mydata}\OperatorTok{$}\NormalTok{DATE), }\DataTypeTok{origin =} \StringTok{'1970-01-01'}\NormalTok{, }\DataTypeTok{tz =} \StringTok{'GMT'}\NormalTok{))}

\NormalTok{amounts <-}\StringTok{ }\NormalTok{mydata[}\DecValTok{4}\NormalTok{]}

\NormalTok{totalSupply <-}\StringTok{ }\DecValTok{192443301}
\NormalTok{subUnits <-}\StringTok{ }\DecValTok{18}
\NormalTok{totalAmount <-}\StringTok{ }\NormalTok{totalSupply }\OperatorTok{*}\StringTok{ }\NormalTok{(}\DecValTok{10} \OperatorTok{^}\StringTok{ }\NormalTok{subUnits)}

\KeywordTok{head}\NormalTok{(mydata)}
\end{Highlighting}
\end{Shaded}

\begin{verbatim}
##   FROMNODE  TONODE       DATE  TOKENAMOUNT
## 1       82 1443996 2018-04-24 4.071000e+19
## 2       82 1443997 2018-04-24 2.291000e+19
## 3        5 1443998 2018-04-24 2.297303e+18
## 4  1443999 1444000 2018-04-24 8.740000e+18
## 5       44 1444001 2018-04-24 1.180000e+18
## 6        5 1444002 2018-04-24 3.276959e+20
\end{verbatim}

\subsection{Preprocessing}\label{preprocessing}

The preprocessing step involves removal of fraudulent transactions which
might affect the distribution estimate negatively. The total supply of
the networkbnb token is 192443301 (quoted from etherscan.io) and the
range of subunits for the token is 18 decimal units. Thus any
transaction that attempts to log a value greater than the product of
total supply and subunits is deemed as fraudulent.

The token networkbnb does not have any fraudulent transactions.

\begin{Shaded}
\begin{Highlighting}[]
\NormalTok{temp <-}\StringTok{ }\KeywordTok{which}\NormalTok{(mydata}\OperatorTok{<}\StringTok{ }\NormalTok{totalAmount)}
\CommentTok{#print meta data }
\KeywordTok{message}\NormalTok{(}\StringTok{'Maximum allowed amount : '}\NormalTok{, totalAmount)}
\end{Highlighting}
\end{Shaded}

\begin{verbatim}
## Maximum allowed amount : 1.92443301e+26
\end{verbatim}

\begin{Shaded}
\begin{Highlighting}[]
\NormalTok{count <-}\StringTok{ }\DecValTok{0}
\NormalTok{outliers <-}\StringTok{ }\DecValTok{0}
\ControlFlowTok{for}\NormalTok{( a }\ControlFlowTok{in} \DecValTok{1}\OperatorTok{:}\KeywordTok{nrow}\NormalTok{(amounts))\{}
  \ControlFlowTok{if}\NormalTok{( a }\OperatorTok{>}\StringTok{ }\NormalTok{totalAmount)\{}
\NormalTok{    outliers <-}\StringTok{ }\NormalTok{outliers }\OperatorTok{+}\StringTok{ }\DecValTok{1}
\NormalTok{  \}}
  \ControlFlowTok{else}\NormalTok{\{}
\NormalTok{    count <-}\StringTok{ }\NormalTok{count }\OperatorTok{+}\StringTok{ }\DecValTok{1}
\NormalTok{  \}}
\NormalTok{\}}
\KeywordTok{message}\NormalTok{(}\StringTok{'Number of outliers : '}\NormalTok{,outliers)}
\end{Highlighting}
\end{Shaded}

\begin{verbatim}
## Number of outliers : 0
\end{verbatim}

\begin{Shaded}
\begin{Highlighting}[]
\KeywordTok{message}\NormalTok{(}\StringTok{'Number of valid amounts : '}\NormalTok{,count)}
\end{Highlighting}
\end{Shaded}

\begin{verbatim}
## Number of valid amounts : 357142
\end{verbatim}

\subsection{Calculating and plotting selling
frequency}\label{calculating-and-plotting-selling-frequency}

\begin{verbatim}
## Using freq as weighting variable
\end{verbatim}

\begin{verbatim}
##   Users_Count Sell_Count
## 1           1      16575
## 2           2       3962
## 3           3       2115
## 4           4       1284
## 5           5        870
## 6           6        702
\end{verbatim}

\includegraphics{Report_files/figure-latex/pressure-1.pdf}

\begin{verbatim}
## summary statistics
## ------
## min:  65   max:  34809 
## median:  399 
## mean:  994.8245 
## estimated sd:  2931.883 
## estimated skewness:  7.48285 
## estimated kurtosis:  68.98622
\end{verbatim}

\subsection{Approximating the selling
distributions}\label{approximating-the-selling-distributions}

From the above Cullen and Frey graph we could narrow down our
distribution selection to Weibull, lognormal, gamma and poisson.

\begin{Shaded}
\begin{Highlighting}[]
\NormalTok{distributionFit_Seller_pois <-}\StringTok{ }\KeywordTok{fitdist}\NormalTok{(countFromFf}\OperatorTok{$}\NormalTok{Sell_Count, }\StringTok{"pois"}\NormalTok{, }\DataTypeTok{method =}\StringTok{"mle"}\NormalTok{)}
\NormalTok{distributionFit_Seller_wb <-}\StringTok{ }\KeywordTok{fitdist}\NormalTok{(countFromFf}\OperatorTok{$}\NormalTok{Sell_Count, }\StringTok{"weibull"}\NormalTok{, }\DataTypeTok{method =}\StringTok{"mle"}\NormalTok{)}
\NormalTok{distributionFit_Seller_ln <-}\StringTok{ }\KeywordTok{fitdist}\NormalTok{(countFromFf}\OperatorTok{$}\NormalTok{Sell_Count, }\StringTok{"lnorm"}\NormalTok{, }\DataTypeTok{method =}\StringTok{"mle"}\NormalTok{)}
\NormalTok{distributionFit_Seller_gm <-}\StringTok{ }\KeywordTok{fitdist}\NormalTok{(countFromFf}\OperatorTok{$}\NormalTok{Sell_Count, }\StringTok{"gamma"}\NormalTok{ ,}\DataTypeTok{method=}\StringTok{"mme"}\NormalTok{)}
\NormalTok{distributionFit_Seller_wb}
\end{Highlighting}
\end{Shaded}

\begin{verbatim}
## Fitting of the distribution ' weibull ' by maximum likelihood 
## Parameters:
##          estimate  Std. Error
## shape   0.7719378  0.02515483
## scale 774.1209761 56.42896598
\end{verbatim}

\begin{Shaded}
\begin{Highlighting}[]
\KeywordTok{plot}\NormalTok{(distributionFit_Seller_wb)}
\end{Highlighting}
\end{Shaded}

\includegraphics{Report_files/figure-latex/unnamed-chunk-3-1.pdf}

\begin{Shaded}
\begin{Highlighting}[]
\NormalTok{distributionFit_Seller_pois}
\end{Highlighting}
\end{Shaded}

\begin{verbatim}
## Fitting of the distribution ' pois ' by maximum likelihood 
## Parameters:
##        estimate Std. Error
## lambda 994.8245   1.664616
\end{verbatim}

\begin{Shaded}
\begin{Highlighting}[]
\KeywordTok{plot}\NormalTok{(distributionFit_Seller_pois)}
\end{Highlighting}
\end{Shaded}

\includegraphics{Report_files/figure-latex/unnamed-chunk-3-2.pdf}

\begin{Shaded}
\begin{Highlighting}[]
\NormalTok{distributionFit_Seller_ln}
\end{Highlighting}
\end{Shaded}

\begin{verbatim}
## Fitting of the distribution ' lnorm ' by maximum likelihood 
## Parameters:
##          estimate Std. Error
## meanlog 6.1329835 0.04809244
## sdlog   0.9112216 0.03400630
\end{verbatim}

\begin{Shaded}
\begin{Highlighting}[]
\KeywordTok{plot}\NormalTok{(distributionFit_Seller_ln)}
\end{Highlighting}
\end{Shaded}

\includegraphics{Report_files/figure-latex/unnamed-chunk-3-3.pdf}

\begin{Shaded}
\begin{Highlighting}[]
\NormalTok{distributionFit_Seller_gm}
\end{Highlighting}
\end{Shaded}

\begin{verbatim}
## Fitting of the distribution ' gamma ' by matching moments 
## Parameters:
##           estimate
## shape 0.1154545321
## rate  0.0001160552
\end{verbatim}

\begin{Shaded}
\begin{Highlighting}[]
\KeywordTok{plot}\NormalTok{(distributionFit_Seller_gm)}
\end{Highlighting}
\end{Shaded}

\includegraphics{Report_files/figure-latex/unnamed-chunk-3-4.pdf}

\subsection{Calculating the buying
frequency}\label{calculating-the-buying-frequency}

\begin{Shaded}
\begin{Highlighting}[]
\NormalTok{countToDf <-}\StringTok{ }\KeywordTok{count}\NormalTok{(mydata, }\StringTok{"TONODE"}\NormalTok{)}
\NormalTok{countToFf <-}\StringTok{ }\KeywordTok{count}\NormalTok{(countToDf, }\StringTok{"freq"}\NormalTok{)}
\end{Highlighting}
\end{Shaded}

\begin{verbatim}
## Using freq as weighting variable
\end{verbatim}

\begin{Shaded}
\begin{Highlighting}[]
\KeywordTok{colnames}\NormalTok{(countToFf) <-}\StringTok{ }\KeywordTok{c}\NormalTok{(}\StringTok{"Users_Count"}\NormalTok{, }\StringTok{"Buy_Count"}\NormalTok{)}
\KeywordTok{head}\NormalTok{(countToFf)}
\end{Highlighting}
\end{Shaded}

\begin{verbatim}
##   Users_Count Buy_Count
## 1           1    252994
## 2           2     56706
## 3           3     16029
## 4           4      5184
## 5           5      2240
## 6           6      1452
\end{verbatim}

\begin{Shaded}
\begin{Highlighting}[]
\KeywordTok{descdist}\NormalTok{(countToFf}\OperatorTok{$}\NormalTok{Buy_Count, }\DataTypeTok{boot=}\DecValTok{500}\NormalTok{)}
\end{Highlighting}
\end{Shaded}

\includegraphics{Report_files/figure-latex/unnamed-chunk-4-1.pdf}

\begin{verbatim}
## summary statistics
## ------
## min:  24   max:  252994 
## median:  117.5 
## mean:  6157.621 
## estimated sd:  33890.53 
## estimated skewness:  7.076031 
## estimated kurtosis:  54.78311
\end{verbatim}

\subsection{Approximating the buying
distributions}\label{approximating-the-buying-distributions}

\begin{Shaded}
\begin{Highlighting}[]
\NormalTok{distributionFit_Buyer_pois <-}\StringTok{ }\KeywordTok{fitdist}\NormalTok{(countToFf}\OperatorTok{$}\NormalTok{Buy_Count, }\StringTok{"pois"}\NormalTok{, }\DataTypeTok{method =}\StringTok{"mle"}\NormalTok{)}
\NormalTok{distributionFit_Buyer_wb <-}\StringTok{ }\KeywordTok{fitdist}\NormalTok{(countToFf}\OperatorTok{$}\NormalTok{Buy_Count, }\StringTok{"weibull"}\NormalTok{, }\DataTypeTok{method =}\StringTok{"mle"}\NormalTok{)}
\NormalTok{distributionFit_Buyer_ln <-}\StringTok{ }\KeywordTok{fitdist}\NormalTok{(countToFf}\OperatorTok{$}\NormalTok{Buy_Count, }\StringTok{"lnorm"}\NormalTok{, }\DataTypeTok{method =}\StringTok{"mle"}\NormalTok{)}
\NormalTok{distributionFit_Buyer_gm <-}\StringTok{ }\KeywordTok{fitdist}\NormalTok{(countToFf}\OperatorTok{$}\NormalTok{Buy_Count, }\StringTok{"gamma"}\NormalTok{, }\DataTypeTok{method =}\StringTok{"mme"}\NormalTok{)}

\NormalTok{distributionFit_Buyer_pois}
\end{Highlighting}
\end{Shaded}

\begin{verbatim}
## Fitting of the distribution ' pois ' by maximum likelihood 
## Parameters:
##        estimate Std. Error
## lambda 6157.621   10.26635
\end{verbatim}

\begin{Shaded}
\begin{Highlighting}[]
\KeywordTok{plot}\NormalTok{(distributionFit_Buyer_pois)}
\end{Highlighting}
\end{Shaded}

\includegraphics{Report_files/figure-latex/unnamed-chunk-5-1.pdf}

\begin{Shaded}
\begin{Highlighting}[]
\NormalTok{distributionFit_Buyer_wb}
\end{Highlighting}
\end{Shaded}

\begin{verbatim}
## Fitting of the distribution ' weibull ' by maximum likelihood 
## Parameters:
##          estimate   Std. Error
## shape   0.3913615   0.03285488
## scale 595.1275712 213.11455385
\end{verbatim}

\begin{Shaded}
\begin{Highlighting}[]
\KeywordTok{plot}\NormalTok{(distributionFit_Buyer_wb)}
\end{Highlighting}
\end{Shaded}

\includegraphics{Report_files/figure-latex/unnamed-chunk-5-2.pdf}

\begin{Shaded}
\begin{Highlighting}[]
\NormalTok{distributionFit_Buyer_ln}
\end{Highlighting}
\end{Shaded}

\begin{verbatim}
## Fitting of the distribution ' lnorm ' by maximum likelihood 
## Parameters:
##         estimate Std. Error
## meanlog 5.323868  0.2432331
## sdlog   1.852408  0.1719915
\end{verbatim}

\begin{Shaded}
\begin{Highlighting}[]
\KeywordTok{plot}\NormalTok{(distributionFit_Buyer_ln)}
\end{Highlighting}
\end{Shaded}

\includegraphics{Report_files/figure-latex/unnamed-chunk-5-3.pdf}

\begin{Shaded}
\begin{Highlighting}[]
\NormalTok{distributionFit_Buyer_gm}
\end{Highlighting}
\end{Shaded}

\begin{verbatim}
## Fitting of the distribution ' gamma ' by matching moments 
## Parameters:
##           estimate
## shape 3.359095e-02
## rate  5.455183e-06
\end{verbatim}

\begin{Shaded}
\begin{Highlighting}[]
\KeywordTok{plot}\NormalTok{(distributionFit_Buyer_gm)}
\end{Highlighting}
\end{Shaded}

\includegraphics{Report_files/figure-latex/unnamed-chunk-5-4.pdf}

\subsection{Conclusion}\label{conclusion}

From the above graph estimates, both buy and sell frequency for our
dataset follows LOG-NORMAL distribution as the standard error is least
and the emperical distribution curve follows the theoritical
distribution curve most accurately.

\subsection{Study 2 :}\label{study-2}

We are trying to find the correlation between the unique number if
buyers each day to the token opening price for the day.

\subsection{Read the price file}\label{read-the-price-file}

Price file contains details of the open, clase, max and min price for
the token foe each day

\begin{Shaded}
\begin{Highlighting}[]
\NormalTok{pricefile <-}\StringTok{'bnb.txt'}
\NormalTok{col_names <-}\StringTok{ }\KeywordTok{c}\NormalTok{(}\StringTok{"Date"}\NormalTok{,}\StringTok{"Open"}\NormalTok{,}\StringTok{"High"}\NormalTok{,}\StringTok{"Low"}\NormalTok{,}\StringTok{"Close"}\NormalTok{,}\StringTok{"Volume"}\NormalTok{,}\StringTok{"MarketCap"}\NormalTok{)}
\NormalTok{myPrices <-}\StringTok{ }\KeywordTok{read.csv}\NormalTok{( pricefile , }\DataTypeTok{header =} \OtherTok{TRUE}\NormalTok{, }\DataTypeTok{sep =} \StringTok{"}\CharTok{\textbackslash{}t}\StringTok{"}\NormalTok{, }\DataTypeTok{dec =} \StringTok{"."}\NormalTok{, }\DataTypeTok{col.names =}\NormalTok{ col_names)}
\NormalTok{myPrices}\OperatorTok{$}\NormalTok{Date <-}\StringTok{ }\KeywordTok{format}\NormalTok{(}\KeywordTok{as.Date}\NormalTok{(myPrices}\OperatorTok{$}\NormalTok{Date, }\DataTypeTok{format =} \StringTok{"%m/%d/%Y"}\NormalTok{), }\StringTok{"%Y-%m-%d"}\NormalTok{)}
\KeywordTok{head}\NormalTok{(myPrices)}
\end{Highlighting}
\end{Shaded}

\begin{verbatim}
##         Date  Open  High   Low Close     Volume     MarketCap
## 1 2018-07-04 14.23 14.33 13.91 14.01 37,043,700 1,622,370,000
## 2 2018-07-03 14.56 14.78 14.08 14.17 60,657,300 1,660,830,000
## 3 2018-07-02 14.40 14.82 14.06 14.57 55,614,000 1,641,930,000
## 4 2018-07-01 14.68 14.69 14.14 14.40 38,434,400 1,673,690,000
## 5 2018-06-30 14.55 15.18 14.29 14.66 59,676,900 1,659,200,000
## 6 2018-06-29 14.17 14.65 13.78 14.51 52,784,600 1,616,460,000
\end{verbatim}

\subsection{Studying distribution of the opening price
.}\label{studying-distribution-of-the-opening-price-.}

The see the pattern for opening price values each day for BNB token. We
do not see any outliers in this data.

\begin{Shaded}
\begin{Highlighting}[]
\NormalTok{timePrices <-}\StringTok{ }\KeywordTok{subset}\NormalTok{(myPrices, }\DataTypeTok{select=}\KeywordTok{c}\NormalTok{(}\StringTok{"Date"}\NormalTok{,}\StringTok{"Open"}\NormalTok{))}
\NormalTok{timePrices}\OperatorTok{$}\NormalTok{Date <-}\StringTok{ }\KeywordTok{as.Date}\NormalTok{(timePrices}\OperatorTok{$}\NormalTok{Date, }\StringTok{"%Y-%m-%d"}\NormalTok{)}
\NormalTok{timePrices <-}\StringTok{ }\KeywordTok{unique}\NormalTok{(timePrices)}
\KeywordTok{summary}\NormalTok{(timePrices)}
\end{Highlighting}
\end{Shaded}

\begin{verbatim}
##       Date                 Open         
##  Min.   :2017-07-25   Min.   : 0.09972  
##  1st Qu.:2017-10-19   1st Qu.: 1.58000  
##  Median :2018-01-13   Median : 8.94000  
##  Mean   :2018-01-13   Mean   : 7.75033  
##  3rd Qu.:2018-04-09   3rd Qu.:13.14000  
##  Max.   :2018-07-04   Max.   :22.77000
\end{verbatim}

\begin{Shaded}
\begin{Highlighting}[]
\KeywordTok{plot}\NormalTok{(timePrices}\OperatorTok{$}\NormalTok{Date, timePrices}\OperatorTok{$}\NormalTok{Open, }\DataTypeTok{main =} \StringTok{"Opening prices VS date"}\NormalTok{, }\DataTypeTok{xlab =} \StringTok{"Date"}\NormalTok{, }\DataTypeTok{ylab=}\StringTok{"Open price"}\NormalTok{)}
\end{Highlighting}
\end{Shaded}

\includegraphics{Report_files/figure-latex/unnamed-chunk-7-1.pdf}

\subsection{Studying the distribution of number of unique buyers each
day.}\label{studying-the-distribution-of-number-of-unique-buyers-each-day.}

We see outliers in this data.

\begin{Shaded}
\begin{Highlighting}[]
\NormalTok{timeBuyFreq <-}\StringTok{ }\KeywordTok{ddply}\NormalTok{(mydata, .(DATE), mutate, }\DataTypeTok{count =} \KeywordTok{length}\NormalTok{(}\KeywordTok{unique}\NormalTok{(TONODE)))}
\NormalTok{timeBuyFreq <-}\StringTok{ }\KeywordTok{subset}\NormalTok{(timeBuyFreq, }\DataTypeTok{select=}\KeywordTok{c}\NormalTok{(}\StringTok{"DATE"}\NormalTok{, }\StringTok{"count"}\NormalTok{))}
\NormalTok{timeBuyFreq}\OperatorTok{$}\NormalTok{DATE <-}\StringTok{ }\KeywordTok{as.Date}\NormalTok{(timeBuyFreq}\OperatorTok{$}\NormalTok{DATE, }\StringTok{"%Y-%m-%d"}\NormalTok{)}
\NormalTok{timeBuyFreq <-}\StringTok{ }\KeywordTok{unique}\NormalTok{(timeBuyFreq)}
\KeywordTok{summary}\NormalTok{(timeBuyFreq)}
\end{Highlighting}
\end{Shaded}

\begin{verbatim}
##       DATE                count          
##  Min.   :2017-07-07   Min.   :     3.00  
##  1st Qu.:2017-09-20   1st Qu.:    53.75  
##  Median :2017-12-05   Median :   148.00  
##  Mean   :2017-12-05   Mean   :  1025.48  
##  3rd Qu.:2018-02-19   3rd Qu.:   236.50  
##  Max.   :2018-05-06   Max.   :117595.00
\end{verbatim}

\begin{Shaded}
\begin{Highlighting}[]
\NormalTok{outliers <-}\StringTok{ }\KeywordTok{boxplot}\NormalTok{(timeBuyFreq}\OperatorTok{$}\NormalTok{count, }\DataTypeTok{main=}\StringTok{"Unique buyer count distribution"}\NormalTok{, }\DataTypeTok{ylab=}\StringTok{"unique buyer count"}\NormalTok{)}\OperatorTok{$}\NormalTok{out}
\end{Highlighting}
\end{Shaded}

\includegraphics{Report_files/figure-latex/unnamed-chunk-8-1.pdf}

We see the summary of the outliers and plot the data with and without
the outliers.

\begin{Shaded}
\begin{Highlighting}[]
\KeywordTok{summary}\NormalTok{(outliers)}
\end{Highlighting}
\end{Shaded}

\begin{verbatim}
##    Min. 1st Qu.  Median    Mean 3rd Qu.    Max. 
##     570    1118    1746   10785    3600  117595
\end{verbatim}

\begin{Shaded}
\begin{Highlighting}[]
\KeywordTok{plot}\NormalTok{( timeBuyFreq}\OperatorTok{$}\NormalTok{DATE, timeBuyFreq}\OperatorTok{$}\NormalTok{count ,}\DataTypeTok{ylim=}\KeywordTok{c}\NormalTok{(}\DecValTok{0}\NormalTok{, }\DecValTok{633}\NormalTok{), }\DataTypeTok{main =} \StringTok{"Unique buyer count VS date"}\NormalTok{, }\DataTypeTok{xlab =} \StringTok{"Date"}\NormalTok{, }\DataTypeTok{ylab=}\StringTok{"Unique buyer count"}\NormalTok{)}
\end{Highlighting}
\end{Shaded}

\includegraphics{Report_files/figure-latex/unnamed-chunk-9-1.pdf}

\subsection{Combine opening price and unique buyer count for each
day}\label{combine-opening-price-and-unique-buyer-count-for-each-day}

We remove the outliers are merge the price and buyer counts to find the
pearson correlation between the two fields with each day being a layer.

\begin{Shaded}
\begin{Highlighting}[]
\NormalTok{remove_outliers <-}\StringTok{ }\ControlFlowTok{function}\NormalTok{(x, }\DataTypeTok{na.rm =} \OtherTok{TRUE}\NormalTok{, ...) \{}
\NormalTok{  qnt <-}\StringTok{ }\KeywordTok{quantile}\NormalTok{(x, }\DataTypeTok{probs=}\KeywordTok{c}\NormalTok{(.}\DecValTok{25}\NormalTok{, .}\DecValTok{75}\NormalTok{), }\DataTypeTok{na.rm =}\NormalTok{ na.rm, ...)}
\NormalTok{  H <-}\StringTok{ }\FloatTok{2.5} \OperatorTok{*}\StringTok{ }\KeywordTok{IQR}\NormalTok{(x, }\DataTypeTok{na.rm =}\NormalTok{ na.rm)}
\NormalTok{  y <-}\StringTok{ }\NormalTok{x}
\NormalTok{  y[x }\OperatorTok{<}\StringTok{ }\NormalTok{(qnt[}\DecValTok{1}\NormalTok{] }\OperatorTok{-}\StringTok{ }\NormalTok{H)] <-}\StringTok{ }\OtherTok{NA}
\NormalTok{  y[x }\OperatorTok{>}\StringTok{ }\NormalTok{(qnt[}\DecValTok{2}\NormalTok{] }\OperatorTok{+}\StringTok{ }\NormalTok{H)] <-}\StringTok{ }\OtherTok{NA}
\NormalTok{  y}
\NormalTok{\}}

\NormalTok{priceSellForEachDay <-}\StringTok{ }\KeywordTok{merge}\NormalTok{(}\DataTypeTok{x=}\NormalTok{timePrices, }\DataTypeTok{y=}\NormalTok{timeBuyFreq, }\DataTypeTok{by.x=}\KeywordTok{c}\NormalTok{(}\StringTok{"Date"}\NormalTok{), }\DataTypeTok{by.y =} \KeywordTok{c}\NormalTok{(}\StringTok{"DATE"}\NormalTok{))}
\KeywordTok{head}\NormalTok{(priceSellForEachDay)}
\end{Highlighting}
\end{Shaded}

\begin{verbatim}
##         Date     Open count
## 1 2017-07-25 0.115203    16
## 2 2017-07-26 0.105893     8
## 3 2017-07-27 0.105108    16
## 4 2017-07-28 0.107632    13
## 5 2017-07-29 0.104782     3
## 6 2017-07-30 0.107935     3
\end{verbatim}

\begin{Shaded}
\begin{Highlighting}[]
\NormalTok{newSet <-}\StringTok{ }\KeywordTok{remove_outliers}\NormalTok{(priceSellForEachDay}\OperatorTok{$}\NormalTok{count)}
\NormalTok{maxCount =}\StringTok{ }\KeywordTok{max}\NormalTok{(newSet[}\KeywordTok{complete.cases}\NormalTok{(newSet)])}
\NormalTok{minCount =}\StringTok{ }\KeywordTok{min}\NormalTok{(newSet[}\KeywordTok{complete.cases}\NormalTok{(newSet)])}
\NormalTok{priceSellForEachDay <-}\StringTok{ }\KeywordTok{subset}\NormalTok{(priceSellForEachDay, count}\OperatorTok{<}\NormalTok{maxCount }\OperatorTok{&}\StringTok{ }\NormalTok{count}\OperatorTok{>}\NormalTok{minCount)}
\KeywordTok{cor}\NormalTok{(priceSellForEachDay}\OperatorTok{$}\NormalTok{Open, priceSellForEachDay}\OperatorTok{$}\NormalTok{count, }\DataTypeTok{method=}\KeywordTok{c}\NormalTok{(}\StringTok{"pearson"}\NormalTok{))}
\end{Highlighting}
\end{Shaded}

\begin{verbatim}
## [1] 0.7335533
\end{verbatim}

\subsection{Conclusion}\label{conclusion-1}

We find a very strong positive correlation between the number of people
buying BNB token in a day to the price of the token that day. So we
combine both plots to visualize the correlation.

\begin{Shaded}
\begin{Highlighting}[]
\CommentTok{#' Create the two plots.}
\NormalTok{p1 <-}\StringTok{ }\KeywordTok{ggplot}\NormalTok{(priceSellForEachDay, }\KeywordTok{aes}\NormalTok{(Date, count)) }\OperatorTok{+}\StringTok{ }\KeywordTok{geom_line}\NormalTok{() }\OperatorTok{+}\StringTok{ }\KeywordTok{theme_minimal}\NormalTok{() }\OperatorTok{+}\StringTok{ }
\StringTok{      }\KeywordTok{theme}\NormalTok{(}\DataTypeTok{axis.title.x =} \KeywordTok{element_blank}\NormalTok{(), }\DataTypeTok{axis.text.x =} \KeywordTok{element_blank}\NormalTok{())}
\NormalTok{p2 <-}\StringTok{ }\KeywordTok{ggplot}\NormalTok{(priceSellForEachDay,}\KeywordTok{aes}\NormalTok{(Date, Open)) }\OperatorTok{+}\StringTok{ }\KeywordTok{geom_bar}\NormalTok{(}\DataTypeTok{stat=}\StringTok{"identity"}\NormalTok{) }\OperatorTok{+}\StringTok{ }\KeywordTok{theme_minimal}\NormalTok{() }\OperatorTok{+}\StringTok{ }
\StringTok{      }\KeywordTok{theme}\NormalTok{(}\DataTypeTok{axis.title.x =} \KeywordTok{element_blank}\NormalTok{(),}\DataTypeTok{axis.text.x =} \KeywordTok{element_text}\NormalTok{(}\DataTypeTok{angle=}\DecValTok{90}\NormalTok{))}
\KeywordTok{grid.newpage}\NormalTok{()}
\KeywordTok{grid.draw}\NormalTok{(}\KeywordTok{rbind}\NormalTok{(}\KeywordTok{ggplotGrob}\NormalTok{(p1), }\KeywordTok{ggplotGrob}\NormalTok{(p2), }\DataTypeTok{size =} \StringTok{"last"}\NormalTok{))}
\end{Highlighting}
\end{Shaded}

\includegraphics{Report_files/figure-latex/unnamed-chunk-11-1.pdf}


\end{document}
